%%%%%%%%%%%%%%%%%%%%%%%%%%%%%%%%%%%%%%%%%%%%%%%% Preamble %%%%%%%%%%%%%%%%%%%%%%%%%%%%%%%%%%%%%%%%%%%%%%%%
% 메타 정보 입력 섹션
\documentclass{article} % 문서 타입 정의

\usepackage{amsmath, geometry} % 사용할 패키지 import
\geometry{
    a4paper,
    left=25.4mm,
    right=25.4mm,
    top=25.4mm,
    bottom=25.4mm,
    } % 문서 크기 및 여백 설정
\linespread{1.3} % 줄 간 간격 조절 ; default : 1.0

\title{\vspace{0cm} Assignment \#1: Review of Numerical Methods \\ and Basics of Optimization} % 제목
% \author{Course \hfill: Numerical Optimization \\ Name : Chiyoung Kwon % 내 신상
%  \\ ID : 261258263 \\ Department : Mechanical Engineering \\ Program : PhD}
\date{} % 작성일자 ; 비워져있으면 출력 안됨

%%%%%%%%%%%%%%%%%%%%%%%%%%%%%%%%%%%%%%%%%%%%%%%% Document %%%%%%%%%%%%%%%%%%%%%%%%%%%%%%%%%%%%%%%%%%%%%%%%
% 본문 섹션
\begin{document} % 문서 본문 시작
\maketitle{ % 위에서 쓴 제목, 내 신상, 작성일자 실제로 출력시키는 명령어
    \begin{flushright} % 오른쪽 정렬
        {
        \vspace{-1.5cm}
        Sep 24th, 2025 \\
        \vspace{2mm}
        Course : Numerical Optimization \\ Name : Chiyoung Kwon \\ ID : 261258263 \\
        Department : Mechanical Engineering \\ Program : PhD
        }
    \end{flushright}
    }
{
    \noindent 1. Consider the following function and do the following (by hand): \\
    $ f(x) = 2x_1^2 - 3x_2^2 + 4x_1x_2 + (x_3 + 2)^2 + 4x_1 $ \\
    (a) What are the gradient and Hessian of $ f(x) $? \\
    (b) What are the stationary point(s) of $ f(x) $? \\
    (c) Is the Hessian positive-definite? \\

    \noindent 2. Let f(x) = sin(x) be a function that you are interested in optimizing. Please answer the following
    questions completely: \\
    (a) What are the necessary conditions for a solution to be an optimum of $ f(x) $? \\
    (b) Using the necessary conditions obtained in (a), and considering the interval $ 0 \leq x \leq 2\pi $, obtain
    the stationary point(s)? \\
    (c) Confirm whether the above point(s) are inflection points, maxima, or minima. If they are maxi-
    mum (or minimum) points, are they global maximum (or minimum) in the given interval? \\
    (d) Plot the function $ sin(x) $ over the interval $ 0 \leq x \leq 2\pi $. Show all the stationary points on it, and
    label them appropriately (maximum, minimum, or inflection). \\
    
    \noindent 3. Consider the single variable function $ f(x) = e - ax^2 $, where a is a constant. This function is often used as
    a ”radial basis function” for function approximation. Please answer the following questions completely: \\
    (a) Is the point $ x = 0 $ a stationary point for (i) $ a > 0 $, and (ii) $ a < 0 $. What happens if $ a = 0 $? Is
    $ x = 0 $ still a stationary point? \\
    (b) If $ x = 0 $ is a stationary point, classify it as a minimum, maximum, or an inflection point for (i)
    $ a > 0 $ ,(ii) $ a < 0 $, and (iii) $ a = 0 $. \\
    (c) Prepare a plot of f(x) for $ a = 1 $, $ a = 2 $, and $ a = 3 $. Plot all three curves on the same figure. By
    observing the plot, do you think $ f(x) = e - ax^2 $, $ a > 0 $ has a global minimum? If so, what is the
    value of $ x $ and $ f(x) $ at the minimum? \\

    \noindent 4. Let $ A = \begin{bmatrix} 3 & 4 \\ 2 & 1 \end{bmatrix} $ \\ % 행렬 쓰는 법
    (a) Use the definition to determine whether $ \begin{bmatrix} -\pi \\ \pi \end{bmatrix} $ and $ \begin{bmatrix} 1 \\ 2 \end{bmatrix} $
    are eigenvectors. of $ A $ associated with $ \lambda = -1 $. \\
    (b) Is either of the given eigenvectors of A associated with $ \lambda = 5 $? \\
    (c) What is the point of this exercise? \\

    \noindent 5. Use the Bisection, Fixed-Point, Newton’s, and Secant methods to find solutions accurate to within 
    $ 10^{-5} $ for the following problems: \\
    (a) $ x^2 - 4x + 4 - ln(x) = 0 $ for $ 1 \leq x \leq 2 $ and $ 2 \leq x \leq 4 $ \\
    (b) $ x + 1 - 2sin(\pi x) = 0 $ for $ 0 \leq x \leq 0.5 $ and $ 0.5 \leq x \leq 1 $ \\
    Write a code to solve the above problems using the specified methods. Provide a plot illustrating the
    convergence of the error versus the number of iterations. For the fixed-point, Newton’s, and Secant
    methods set $ x_0 $ to be the minimum point for the specified range. In addition to the plot, show a table
    with four entries of the values of $ x, f(x) $ and the error $ f(x) $. You may treat $ x $ as $ \bar{x} $ in these cases. Out
    of the four entries, provide the initial value, the final values and two intermediary values during the
    convergence of the algorithm. \\

    \noindent 6. Let $ f(x, y) = x^3 - x + y^3 - y $ \\
    (a) Graph the surface $ z = f(x, y) $. \\
    (b) Verify that the complete list of critical points of $ f $ is
    \( % 문장 안에 수식 넣는 법 ; inline 수식 방식
    \left(-\frac{1}{\sqrt{3}}, -\frac{1}{\sqrt{3}}\right),\ % 수식에 맞춰서 괄호 크기 자동 조정 ; 분수 표기
    \left(-\frac{1}{\sqrt{3}}, \frac{1}{\sqrt{3}}\right),\ 
    \left(\frac{1}{\sqrt{3}}, -\frac{1}{\sqrt{3}}\right),\ 
    \left(\frac{1}{\sqrt{3}}, \frac{1}{\sqrt{3}}\right)
    \)
    (c) Calculate the Hessian matrix? \\
    (d) Fill in the following table. \\
    \begin{tabular}{|c|c|c|c|}
        \hline
        Critical points & Hessian at $ (x, y) $ & Eigenvalues of Hessian at $ (x, y) $ & Concavity at $ (x, y) $ \\
        \hline
        $ \left(-\frac{1}{\sqrt{3}}, -\frac{1}{\sqrt{3}}\right) $ & & & \\
        \hline
        $ \left(-\frac{1}{\sqrt{3}}, \frac{1}{\sqrt{3}}\right) $ & & & \\
        \hline
        $ \left(\frac{1}{\sqrt{3}}, -\frac{1}{\sqrt{3}}\right) $ & & & \\
        \hline
        $ \left(\frac{1}{\sqrt{3}}, \frac{1}{\sqrt{3}}\right) $ & & & \\
        \hline
    \end{tabular} \\
    (e) Examine the table carefully and explain how eigenvalues of Hessian matrices can help you classify
    the concavity of the surface at each critical point. \\

    \noindent 7. Find the critical points of $ f(x, y) = x^2 + y^3 - x^2y + xy^2 $ and classify them all by using the eigenvalues
    of the appropriate Hessian matrices.

}   

\end{document}