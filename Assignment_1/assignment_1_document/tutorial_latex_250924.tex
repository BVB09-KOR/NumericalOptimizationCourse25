% ※ 명령어 사용 시 백슬래쉬(\) 쓰고 명령어 입력 %
% ※ 명령어 사용 시 arg 받을 수도 있고 안 받을 수도 있음. %
% ※ arg 입력 시
% \command{arg} ... 필수 arg
% \command[arg] ... 선택 arg

%%%%%%%%%%%%%%%%%%%%% Preamble 영역(메타데이터 입력하는 곳) %%%%%%%%%%%%%%%%%%%%%
% 문서 타입 선언 %
% e.g. article(논문, 보고서, 과제 ; section 기반), report(학위논문, 기술 보고서 ; chapter 기반)
% e.g. beamer(ppt슬라이드 느낌), proc(학회 논문집 느낌)
\documentclass{article}

% 패키지 import %
\usepackage{lipsum}
 
% 메타정보(제목, 저자, 작성일) 입력 %
\title{\textbf{The First  \LaTeX  Article}}
\author{\textbf{Author Name} \\ Department Name, University Name}

%%%%%%%%%%%%%%%%%%%%% Document 영역 %%%%%%%%%%%%%%%%%%%%%
%% 본격적인 문서 작성 시작
\begin{document} % document라는 'environment'로의 진입 명령어
    \maketitle % 메타정보(제목, 저자, 작성일) 섹션에서 입력한 정보들을 실제로 출력하는 명령어. 안 쓰면 출력 안 됨.
    \begin{abstract} % document env 내 abstract라는 env로의 진입 명령어
        \lipsum[1] % lipsum 패키지 사용 명령어. []는 lipsum이라는 명령어를 씀에 있어서 optional arg라는 의미.
    \end{abstract} % abstract env 종료 명령어
    \section{Hello, World!} % setion 타이틀 작성 명령어. {}는 section이라는 명령어를 씀에 있어 necessary arg라는 의미.
    \lipsum[2-4] % lipsum 패키지 기능 사용

    \begin{tabular}{ c c c } % tabular라는 env는 필수입력 arg가 있음. 
        cell1 & cell2 & cell3 \\ 
        cell4 & cell5 & cell6 \\ 
        cell7 & cell8 & cell9 \\ 
    \end{tabular}
    
    \begin{tabular}{|p{0.9\textwidth}|}
        \hline \\
        {

        }
        \\\\\hline
    \end{tabular}
\end{document} % document 'environment' 종료 명령어
% ※ environment : 특정한 서식으로 문서작성할 수 있게끔 해주는 일종의 작성 '함수'. LaTeX 다운받으면 딸려오는 기본 내장 environment들이 몇몇 있다.
%  각 패키지마다 그들만의 environment(함수)를 가지고 있음.
%  begin과 end로 진입/종료 한다.
%  아래처럼 사용자가 직접 자신만의 environment를 정의할 수도 있다(마치 함수처럼).
\newenvironment{name}[numarg][optarg_default]{begin_def ~ end_def}
% name : 만들 environment 이름
% numarg : 해당 environemtn 실행 시 받을 argument 개수 설정(optional)
% optarg_defaul : 이 environement가 받는 argument에 대한 default 값 설정(optional)
% {begin_def ~ end_def} : 이 environment에서 구현할 기능들 쓰는 공간

%% environment 선언 예시
\newenvironment{boxed}[2][This is a box]
    {\begin{center}
    Argument 1 = #1\\
    \begin{tabular}{|p{0.9\textwidth}|}
    \hline\\
    Argument 2 = #2\\
    }
    { 
    \\\\\hline
    \end{tabular} 
    \end{center}
    }
% 'boxed'라는 이름의 environment 선언
% 이 environment는 2개의 argument를 받음
% 1번째 argument에 대한 default value는 'This is a box'라는 텍스트임
% 본격적인 environment 정의 구간 내에서 1번째 argument를 뜻하는 게 '#1'이고 2번째 argument 뜻하는 게 '#2'임.
% 즉 '#숫자'는 일반 텍스트가 아니니 인지하고 environment를 짜야 함.
% 실제 environment 쓸 때는 \begin{environment명} 왈왈왈 \end{environment명} 하면 됨.